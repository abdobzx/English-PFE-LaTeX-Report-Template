\chapter{Project Overview}
\label{chapter:ProjectOverview}

\section{Requirements Gathering}
\label{sec:RequirementsGathering}

The initial phase of our project involved a thorough requirements gathering process to ensure the CI/CD pipeline would meet Sothema's operational and technical needs for deploying SharePoint Framework (SPFx) solutions. We collaborated with IT stakeholders, including SharePoint developers and administrators.

\subsection{Functional Requirements}
\label{subsec:FunctionalRequirements}

The functional requirements define what the system must do:
\begin{itemize}
    \item \textbf{FR1: Automated Build:} The system shall automatically trigger a build process upon code commit to designated branches in the Azure Repos repository.
    \item \textbf{FR2: SPFx Packaging:} The pipeline must correctly bundle and package the SPFx solution into a \texttt{.sppkg} file.
    \item \textbf{FR3: Versioning:} The \texttt{.sppkg} file and other build artifacts must be versioned consistently.
    \item \textbf{FR4: Automated Unit Testing:} The pipeline shall execute Node.js-based unit tests (e.g., using \texttt{gulp test}) and fail the build if tests do not pass.
    \item \textbf{FR5: Automated End-to-End Testing:} The pipeline shall execute Playwright end-to-end tests against a SharePoint workbench or a designated test site.
    \item \textbf{FR6: Vulnerability Scanning:} The pipeline must scan Docker images and application dependencies for known vulnerabilities (e.g., using Trivy).
    \item \textbf{FR7: Multi-Environment Deployment:} The pipeline must support deployment to multiple SharePoint Online environments: Development (Dev), Quality Assurance (QA), Pre-Production (PreProd), and Production (Prod).
    \item \textbf{FR8: App Catalog Management:} The pipeline shall automate the upload and publishing of the \texttt{.sppkg} file to the SharePoint App Catalog for each environment.
    \item \textbf{FR9: Solution Deployment to Sites:} The pipeline shall automate the installation/update of the SPFx solution on specified target SharePoint sites within each environment.
    \item \textbf{FR10: List Creation (if applicable):} The pipeline should be capable of running post-deployment scripts to create or update SharePoint lists required by the SPFx solution.
    \item \textbf{FR11: Approval Gates:} Manual approval gates must be implemented before deployment to QA, PreProd, and Prod environments.
    \item \textbf{FR12: Notifications:} The pipeline should provide notifications for build/deployment success or failure.
    \item \textbf{FR13: Artifact Storage:} Build artifacts must be stored securely and be accessible for review and deployment.
    \item \textbf{FR14: Rollback Capability:} The release pipeline should facilitate rollback to a previous stable version in case of deployment failure (achieved by re-deploying a previous successful artifact).
\end{itemize}

\subsection{Non-Functional Requirements}
\label{subsec:NonFunctionalRequirements}

Non-functional requirements define the quality attributes of the system:
\begin{itemize}
    \item \textbf{NFR1: Security:}
    \begin{itemize}
        \item Secure management of credentials (e.g., SharePoint Online credentials, Azure AD service principal details) using Azure Key Vault.
        \item Enforcement of least privilege for the service principal used by the pipeline.
        \item Vulnerability scanning policies must fail the build for HIGH or CRITICAL severity vulnerabilities.
    \end{itemize}
    \item \textbf{NFR2: Reliability:} The pipeline must be robust and handle transient errors gracefully (e.g., network issues during deployment, with retry mechanisms if feasible).
    \item \textbf{NFR3: Performance:} The CI pipeline execution time should be optimized (e.g., aiming for completion within 10-15 minutes for a typical SPFx project).
    \item \textbf{NFR4: Maintainability:}
    \begin{itemize}
        \item The pipeline configuration (YAML) must be well-documented and easy to understand.
        \item Scripts used in the pipeline (PowerShell) must be modular and reusable.
    \end{itemize}
    \item \textbf{NFR5: Usability:} The pipeline status, logs, and reports should be easily accessible to developers and IT operations personnel via the Azure DevOps portal.
    \item \textbf{NFR6: Scalability:} The pipeline design should accommodate an increasing number of SPFx solutions and developers over time.
    \item \textbf{NFR7: Traceability:} All pipeline executions, approvals, and deployments must be logged for audit purposes.
    \item \textbf{NFR8: Compliance:} The pipeline should support adherence to Sothema's internal IT governance and security policies.
\end{itemize}

\section{High-Level Architecture}
\label{sec:HighLevelArchitecture}

The CI/CD pipeline was designed using Azure DevOps as the central orchestration platform. The architecture integrates several key components to automate the build, test, scan, and deployment of SPFx solutions.

\begin{figure}[htbp]
    \centering
    % The verbatim block previously here is replaced with a descriptive paragraph.
    % For a visual representation, consider creating a diagram using a dedicated tool
    % and including it as an image (e.g., using the graphicx package and \\includegraphics).
    \parbox{0.8\textwidth}{
        \centering
        \textit{Figure \ref{fig:HighLevelArchitecture} conceptually illustrates the CI/CD pipeline architecture.
        Azure Repos hosts the source code. Azure Pipelines (CI) automates build, test (unit, E2E with Playwright), and vulnerability scanning (Trivy), producing artifacts.
        These artifacts are then deployed by Azure Pipelines (Release) through multiple stages (Dev, QA, PreProd, Prod) to SharePoint Online environments.
        Deployment tasks utilize PowerShell scripts for SharePoint interactions, authenticated via a Service Principal.
        Each SharePoint environment (Dev, QA, PreProd, Prod) includes an App Catalog and target site(s).}
    }
    \caption{High-Level CI/CD Pipeline Architecture}
    \label{fig:HighLevelArchitecture}
\end{figure}

\textbf{Key Components:}
\begin{enumerate}
    \item \textbf{Azure Repos:} Hosts the Git repository containing the SPFx solution source code. It enforces the branch strategy and pull request policies.
    \item \textbf{Azure Pipelines (CI):} Defined by a YAML file (\texttt{azure-pipelines.yml}). This pipeline is triggered by commits to specific branches. It performs:
    \begin{itemize}
        \item Code checkout.
        \item Node.js environment setup and dependency installation.
        \item Unit testing.
        \item SPFx solution bundling and packaging.
        \item (Optional) Building a Docker image if the build/scan process is containerized.
        \item \textbf{Trivy Scan:} Security vulnerability scanning of dependencies or the Docker image.
        \item \textbf{Playwright Tests:} Execution of end-to-end UI tests.
        \item Publishing build artifacts (the \texttt{.sppkg} file, test reports, scan reports, deployment scripts).
    \end{itemize}
    \item \textbf{Build Agents:} Microsoft-hosted agents (e.g., \texttt{ubuntu-latest}) or self-hosted agents execute the CI pipeline jobs. They are equipped with necessary tools like Node.js, npm/yarn, Gulp, Docker client (if used), and browsers for Playwright.
    \item \textbf{Docker (Optional for build, used by Trivy):} While the SPFx build itself doesn't strictly require Docker, Trivy often operates on Docker images or file systems. If a consistent build environment is paramount, the entire build process can be containerized. In our setup, Trivy primarily scans project dependencies and potentially a build-environment Docker image if one were used.
    \item \textbf{Trivy:} An open-source vulnerability scanner integrated into the CI pipeline to detect security issues in application dependencies (via \texttt{package-lock.json} or \texttt{yarn.lock}) and/or Docker images.
    \item \textbf{Playwright:} A Node.js library for browser automation, used to run end-to-end tests against the SPFx components in a headless browser.
    \item \textbf{Azure Pipelines (Release):} Manages the deployment of artifacts to different SharePoint Online environments. It consists of multiple stages (Dev, QA, PreProd, Prod).
    \begin{itemize}
        \item Each stage downloads the necessary artifacts.
        \item It uses \textbf{PowerShell scripts} (leveraging modules like PnP.PowerShell or the Office 365 CLI) to interact with SharePoint Online.
        \item Secure authentication is handled via an Azure AD App Registration (Service Principal) with certificate-based authentication.
    \end{itemize}
    \item \textbf{PowerShell Scripts:} A suite of scripts for tasks like connecting to SharePoint Online, uploading/publishing the \texttt{.sppkg} to the App Catalog, installing the solution on target sites, creating SharePoint lists, and disconnecting.
    \item \textbf{SharePoint Online Sites:} The target environments, each typically having its own App Catalog and site collections for Dev, QA, PreProd, and Prod.
\end{enumerate}

This architecture provides a clear separation of concerns, automates repetitive tasks, and integrates security and quality checks early in the development lifecycle.