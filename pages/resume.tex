\chapter*{Résumé}
\setstretch{1.2}
\normalsize{
\begin{abstract} % Assuming this is for a French abstract, if not, the content should be the same as abstract.tex or removed.
 % Contenu du résumé de l'utilisateur (Français)

 % Le résumé doit être un sommaire concis du rapport, généralement d'environ 250-300 mots. Il doit couvrir les principaux objectifs, la méthodologie, les résultats clés et les conclusions du projet.
 % Ce projet se concentre sur la conception et la mise en œuvre d'un pipeline CI/CD (Intégration Continue/Déploiement Continu) sécurisé et automatisé pour les solutions SharePoint Framework (SPFx) chez Sothema. L'objectif principal est de rationaliser le cycle de vie du développement et du déploiement, d'améliorer la qualité du code et d'intégrer des pratiques de sécurité robustes. Le pipeline s'appuie sur Azure DevOps pour l'orchestration, intégrant des étapes pour les constructions automatisées, les tests complets (unitaires, d'intégration et analyses de sécurité), et les déploiements échelonnés vers les environnements de développement, de pré-production et de production. Les mesures de sécurité clés comprennent les tests statiques de sécurité des applications (SAST) avec SonarQube, les tests dynamiques de sécurité des applications (DAST) avec OWASP ZAP, et l'analyse des dépendances à l'aide de WhiteSource/GitHub Dependabot. La gestion des secrets est assurée par Azure Key Vault. Le pipeline CI/CD mis en œuvre a démontré des améliorations significatives de l'efficacité des déploiements, une réduction des erreurs manuelles et une posture de sécurité renforcée pour les solutions SPFx. Ce rapport détaille la méthodologie du projet, l'architecture du pipeline CI/CD, les outils et technologies employés, les défis rencontrés et les résultats obtenus, ainsi que des recommandations pour les améliorations futures.
\end{abstract}
}

\medskip
{\noindent \textbf{Mots-clés: vos mots-clés ici} }
\setstretch{1}