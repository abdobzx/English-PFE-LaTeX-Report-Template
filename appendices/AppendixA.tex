% User's Section 14.1 Content

% Full CI Pipeline YAML (for Azure DevOps)
\chapter{Full CI Pipeline YAML (\texttt{azure-pipelines.yml})}
\label{appendix:CIPipelineYAML}

\begin{verbatim}
# azure-pipelines.yml
# CI Pipeline for SPFx Solutions at Sothema

trigger:
  branches:
    include:
      - Dev
      - QA
      - PreProd
      - Prod
  paths:
    include:
      - 'spfx-solution/*' # Assuming SPFx code is in a subfolder 'spfx-solution'
      - 'scripts/deployment/*'
    exclude:
      - '*.md' # Exclude markdown file changes from triggering full CI

pr:
  branches:
    include:
      - Dev
      - QA
      - PreProd
      - Prod
  paths:
    include:
      - 'spfx-solution/*'
    exclude:
      - '*.md'

pool:
  vmImage: 'ubuntu-latest'

variables:
  # General variables
  nodeVersion: '16.x' # Ensure this Node.js version is compatible with your SPFx version
  projectFolderName: 'spfx-solution' # Folder containing the package.json for the SPFx project
  solutionName: 'your-spfx-solution-name' # Replace with actual solution name for sppkg
  buildPlatform: 'Any CPU'
  buildConfiguration: 'Release'

  # Link to a Variable Group from Azure DevOps Library for secrets
  # Example: group: 'Sothema-SPFx-Secrets' which contains:
  # CLIENTID, TENANTID, THUMBPRINT, PFX_PASSWORD (for the cert)

stages:
- stage: Build_and_Test
  displayName: 'Build, Test, Scan and Package SPFx Solution'
  jobs:
  - job: MainBuild
    displayName: 'SPFx Build, Test, Scan & Package'
    steps:
    - task: NodeTool@0
      inputs:
        versionSpec: $(nodeVersion)
      displayName: 'Install Node.js $(nodeVersion)'

    - script: |
        echo "NPM Version:"
        npm --version
        echo "Node Version:"
        node --version
        cd $(projectFolderName)
        npm ci
      displayName: 'Install npm dependencies'
      workingDirectory: '$(Build.SourcesDirectory)'

    - script: |
        cd $(projectFolderName)
        gulp test
      displayName: 'Run Unit Tests (gulp test)'
      workingDirectory: '$(Build.SourcesDirectory)'

    - script: |
        cd $(projectFolderName)
        gulp bundle --ship
        gulp package-solution --ship
      displayName: 'Bundle and Package SPFx Solution'
      workingDirectory: '$(Build.SourcesDirectory)'

    # Trivy Scan
    # Ensure Trivy is installed or use a marketplace task
    # This example uses Aquasecurity's Trivy task
    - task: Trivy@1
      inputs:
        scanType: 'fs' # Filesystem scan
        scanTarget: '$(Build.SourcesDirectory)/$(projectFolderName)' # Path to scan
        severity: 'HIGH,CRITICAL' # Severities to scan for
        exitCode: '1' # Fail pipeline if vulnerabilities of specified severity are found
        ignoreUnfixed: true # Ignore vulnerabilities that don't have a fix yet
        format: 'table' # Output format
        output: 'trivy-fs-report.txt' # Output file name
        # ignoreFile: '.trivyignore' # Optional: specify path to .trivyignore file
      displayName: 'Run Trivy Vulnerability Scan (Filesystem)'

    # Playwright Tests
    - script: |
        cd $(projectFolderName)
        npx playwright install --with-deps # Install browser binaries
      displayName: 'Install Playwright browsers'
      workingDirectory: '$(Build.SourcesDirectory)'

    - script: |
        cd $(projectFolderName)
        # Assuming 'playwright:ci' script in package.json is like:
        # "playwright:ci": "playwright test --reporter=junit,line,html"
        npm run playwright:ci
      displayName: 'Run Playwright End-to-End Tests'
      workingDirectory: '$(Build.SourcesDirectory)'
      continueOnError: false # Important: Fail the build if E2E tests fail

    # Publish Test Results (Unit Tests)
    # Ensure your test runner (Jest/Mocha) generates JUnit XML reports
    - task: PublishTestResults@2
      inputs:
        testResultsFormat: 'JUnit'
        testResultsFiles: '**/junit.xml' # Adjust path as per your test runner output
        searchFolder: '$(System.DefaultWorkingDirectory)/$(projectFolderName)/temp/tests' # Common for SPFx default
        mergeTestResults: true
        failTaskOnFailedTests: true
      displayName: 'Publish Unit Test Results'
      condition: succeededOrFailed() # Run this even if previous tasks failed, to publish results

    # Publish Test Results (Playwright E2E Tests)
    - task: PublishTestResults@2
      inputs:
        testResultsFormat: 'JUnit'
        testResultsFiles: 'playwright-report/results.xml' # Default Playwright JUnit reporter output path
        searchFolder: '$(System.DefaultWorkingDirectory)/$(projectFolderName)'
        mergeTestResults: true
        failTaskOnFailedTests: true # Fail build if tests in results.xml failed
      displayName: 'Publish Playwright E2E Test Results'
      condition: succeededOrFailed()

    # Prepare SPFx Package Artifact
    - task: CopyFiles@2
      inputs:
        SourceFolder: '$(Build.SourcesDirectory)/$(projectFolderName)/sharepoint/solution'
        Contents: '*.sppkg'
        TargetFolder: '$(Build.ArtifactStagingDirectory)/sppkg'
      displayName: 'Copy SPFx Package to Staging Directory'

    # Prepare PowerShell Scripts Artifact
    - task: CopyFiles@2
      inputs:
        SourceFolder: '$(Build.SourcesDirectory)/scripts/deployment' # Location of your deployment scripts
        Contents: '*.ps1'
        TargetFolder: '$(Build.ArtifactStagingDirectory)/scripts'
      displayName: 'Copy PowerShell Scripts to Staging Directory'

    # Prepare Trivy Report Artifact
    - task: CopyFiles@2
      inputs:
        SourceFolder: '$(System.DefaultWorkingDirectory)' # Assuming trivy-fs-report.txt is at root or adjust
        Contents: 'trivy-fs-report.txt'
        TargetFolder: '$(Build.ArtifactStagingDirectory)/scan-reports'
        overwrite: true
      displayName: 'Copy Trivy Scan Report to Staging Directory'

    # Prepare Playwright HTML Report Artifact
    - task: CopyFiles@2
      inputs:
        SourceFolder: '$(System.DefaultWorkingDirectory)/$(projectFolderName)/playwright-report' # Default Playwright HTML report location
        Contents: '**' # Copy all contents of the HTML report folder
        TargetFolder: '$(Build.ArtifactStagingDirectory)/playwright-html-report'
        overwrite: true
      displayName: 'Copy Playwright HTML Report to Staging Directory'

    # Publish All Staged Artifacts
    - task: PublishBuildArtifacts@1
      inputs:
        PathtoPublish: '$(Build.ArtifactStagingDirectory)'
        ArtifactName: 'drop-$(Build.BuildNumber)' # Include BuildNumber for uniqueness if needed, or just 'drop'
        publishLocation: 'Container'
      displayName: 'Publish Build Artifacts (sppkg, scripts, reports)'
\end{verbatim}

\textbf{Note:} The \texttt{solutionName} variable should be dynamically determined or correctly set if your \texttt{.sppkg} file name is variable. The provided YAML assumes a consistent project structure and script locations. Adjust paths as necessary for your specific project. Remember to create the Variable Group \texttt{Sothema-SPFx-Secrets} in Azure DevOps Library and link it to this pipeline, populating it with \texttt{CLIENTID}, \texttt{TENANTID}, \texttt{THUMBPRINT}, and \texttt{PFX\_PASSWORD}.
