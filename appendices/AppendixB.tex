% User's Section 14.2 Content

% Core PowerShell Deployment Scripts

\chapter{Core PowerShell Deployment Scripts}
\label{appendix:PowerShellScripts}

These are simplified examples. Real scripts would have more robust error handling, logging, and parameterization.

\section{Connect-SharePointOnline.ps1 (Conceptual)}
\resizebox{\textwidth}{!}{
\begin{verbatim}
# This logic is usually embedded within the main deployment script.
# It assumes variables like $ClientId, $TenantId, $CertificateThumbprint, $TenantUrl are available.
# Example snippet:
# Write-Host "Attempting to connect to SharePoint Online: $TenantUrl"
# try {
#   Connect-PnPOnline -Url $TenantUrl -ClientId $ClientId -Tenant $TenantId -Thumbprint $CertificateThumbprint -WarningAction SilentlyContinue
#   Write-Host "Successfully connected to $TenantUrl"
# } catch {
#   Write-Error "Failed to connect to SharePoint: $($_.Exception.Message)"
#   throw # Re-throw to fail the pipeline task
# }
\end{verbatim}
}

\section{Publish-SPFxPackageToCatalog.ps1 (Conceptual)}
\resizebox{\textwidth}{!}{
\begin{verbatim}
# This logic is usually embedded within the main deployment script.
# Assumes $SPPackagePath and $AppCatalogUrl (if site collection app catalog) or tenant context.
# Example snippet:
# Write-Host "Uploading and publishing SPFx package: $SPPackagePath"
# try {
#   # For Tenant App Catalog
#   $app = Add-PnPApp -Path $SPPackagePath -Publish -Overwrite -Scope Tenant
#   Write-Host "SPFx package '$($app.Title)' published to Tenant App Catalog. ID: $($app.Id)"
#
#   # If using Site Collection App Catalog, the command is slightly different:
#   # Connect-PnPOnline -Url $AppCatalogUrl ... (connect to the site collection app catalog site itself)
#   # $app = Add-PnPApp -Path $SPPackagePath -Publish -Overwrite -Scope SiteCollection
#   # Write-Host "SPFx package published to Site Collection App Catalog: $AppCatalogUrl"
# } catch {
#   Write-Error "Failed to publish SPFx package: $($_.Exception.Message)"
#   throw
# }
# Return $app.Id or necessary app identifiers
\end{verbatim}
}

\section{Install-SPFxSolutionToSite.ps1 (Conceptual)}
\resizebox{\textwidth}{!}{
\begin{verbatim}
# This logic is usually embedded within the main deployment script.
# Assumes $app (from publishing step) and $TargetSiteUrl.
# Example snippet:
# Write-Host "Installing/Updating app '$($app.Title)' on site: $TargetSiteUrl"
# try {
#   Install-PnPApp -Identity $app.Id -Site $TargetSiteUrl
#   Write-Host "App installation/update successfully initiated for '$($app.Title)' on $TargetSiteUrl."
# } catch {
#   Write-Error "Failed to install/update app on $TargetSiteUrl: $($_.Exception.Message)"
#   # Consider if this should be a throwing error or just a warning depending on requirements
#   throw
# }
\end{verbatim}
}

\section{Invoke-DeployToEnvironment.ps1 (Master Orchestration Script Example)}
\resizebox{\textwidth}{!}{
\begin{verbatim}
[CmdletBinding()]
param(
    [Parameter(Mandatory=$true)]
    [string]$TenantUrl, # e.g., https://sothemadev.sharepoint.com

    [Parameter(Mandatory=$true)]
    [string]$AppCatalogScope, # 'Tenant' or 'SiteCollection'

    [Parameter(Mandatory=$false)] # Mandatory if AppCatalogScope is 'SiteCollection'
    [string]$AppCatalogUrl, # e.g., https://sothemadev.sharepoint.com/sites/AppCatalog

    [Parameter(Mandatory=$true)]
    [string]$ClientId,

    [Parameter(Mandatory=$true)]
    [string]$TenantId,

    [Parameter(Mandatory=$true)]
    [string]$CertificateThumbprint, # For PnP PowerShell with cert thumbprint

    [Parameter(Mandatory=$false)]
    [string]$CertificatePfxPath, # Alternative: Path to PFX file
    [Parameter(Mandatory=$false)]
    [string]$CertificatePfxPassword, # Alternative: Password for PFX file

    [Parameter(Mandatory=$true)]
    [string]$SPPackagePath, # Path to the .sppkg file

    [Parameter(Mandatory=$false)]
    [string]$TargetSiteUrlsCsv, # Comma-separated list of site URLs to install the app

    [Parameter(Mandatory=$false)]
    [switch]$SkipFeatureDeployment # For Add-PnPApp / Deploy-PnPApp
)

$ErrorActionPreference = "Stop"

try {
    Import-Module PnP.PowerShell -ErrorAction SilentlyContinue
    Write-Host "PnP PowerShell module imported/available."

    # Connection Parameters
    $connectParams = @{
        Url = $TenantUrl
        ClientId = $ClientId
        TenantId = $TenantId
    }
    if (-not [string]::IsNullOrWhiteSpace($CertificateThumbprint)) {
        $connectParams.Thumbprint = $CertificateThumbprint
        Write-Host "Connecting to SharePoint Online: $TenantUrl using Cert Thumbprint."
    } elseif ((-not [string]::IsNullOrWhiteSpace($CertificatePfxPath)) -and (-not [string]::IsNullOrWhiteSpace($CertificatePfxPassword))) {
        $connectParams.CertificatePath = $CertificatePfxPath
        $connectParams.CertificatePassword = $CertificatePfxPassword
        Write-Host "Connecting to SharePoint Online: $TenantUrl using Cert Path."
    } else {
        throw "Either Certificate Thumbprint or Certificate Path/Password must be provided."
    }

    Connect-PnPOnline @connectParams
    Write-Host "Successfully connected to $TenantUrl."

    # Publish to App Catalog
    Write-Host "Publishing SPFx package '$SPPackagePath' with scope '$AppCatalogScope'."
    $addAppParams = @{
        Path = $SPPackagePath
        Publish = $true
        Overwrite = $true
        Scope = $AppCatalogScope # 'Tenant' or 'SiteCollection'
    }
    if ($AppCatalogScope -eq 'SiteCollection') {
        if ([string]::IsNullOrWhiteSpace($AppCatalogUrl)) {
            throw "AppCatalogUrl is required for SiteCollection scope."
        }
        Write-Host "Targeting Site Collection App Catalog at (implicitly or explicitly connected): $AppCatalogUrl"
    }

    $app = Add-PnPApp @addAppParams
    Write-Host "SPFx package '$($app.Title)' (ID: $($app.Id)) uploaded to App Catalog."

    if ($AppCatalogScope -eq 'Tenant') {
        Write-Host "Ensuring app '$($app.Title)' is deployed in Tenant App Catalog."
        Deploy-PnPApp -Identity $app.Id -SkipFeatureDeployment:$SkipFeatureDeployment.IsPresent
        Write-Host "App '$($app.Title)' deployment status checked/triggered in Tenant App Catalog."
    }

    # Install on Target Sites
    if (-not [string]::IsNullOrWhiteSpace($TargetSiteUrlsCsv)) {
        $siteUrls = $TargetSiteUrlsCsv -split ',' | ForEach-Object {$_.Trim()}
        Write-Host "Found $($siteUrls.Count) target site(s) for app installation."

        foreach ($siteUrl in $siteUrls) {
            if (-not ([string]::IsNullOrWhiteSpace($siteUrl))) {
                Write-Host "Installing/Updating app '$($app.Title)' on site: $siteUrl"
                try {
                    Install-PnPApp -Identity $app.Id -Site $siteUrl
                    Write-Host "App installation/update successfully initiated for '$($app.Title)' on $siteUrl."
                } catch {
                    Write-Warning "Failed to install/update app on $siteUrl: $($_.Exception.Message). Continuing with other sites."
                }
            }
        }
    } else {
        Write-Host "No specific target sites provided for installation. App is available in App Catalog."
    }

    Write-Host "Deployment script completed successfully."

} catch {
    Write-Error "Deployment script failed: $($_.Exception.ToString())"
    throw $_ 
} finally {
    Write-Host "Disconnecting from SharePoint Online."
    Disconnect-PnPOnline -ErrorAction SilentlyContinue
}
\end{verbatim}
}
% ...existing code...
